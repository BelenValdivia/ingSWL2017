\chapter{Gestión y Calidad de Proyectos de SL}
\section{¿Qué es una comunidad de SL?}

Una Comunidad de Software Libre es un {\bf grupo de personas que cooperan entre sí en distintas áreas relacionadas con el Software Libre}. Dependiendo del área de la comunidad, sus objetivos van a variar.
\\
\\
Pero todas tienen en común:

\begin{itemize}
     \item El espíritu cooperativo. 
     \item La búsqueda continua del mejoramiento y difusión del software libre y del conocimiento.
     \item Su principal interés es la libertad de los usuarios
     \item Hay un dominio o interés compartido que le da identidad.
     \item La comunidad es creada y mantenida a partir de las interacciones determinadas, por ejemplo, por actividades conjuntas o discusiones.
     \item Existe una práctica compartida, por ejemplo a través del intercambio de buenas prácticas o lecciones aprendidas.
\end{itemize}

La comunidad de código abierto es diversa y esta altamente motivada. Las comunidades se caracterizan por crear y amplificar el efecto de red, donde la colaboración enriquece los recursos y a las organizaciones que las adoptan.
\\
\\
El éxito de una iniciativa de código abierto depende de: {\bf la sensibilización y la adopción del proyecto}. Manteniendo los foros con vida a través de la información, preguntas y desafíos.
%\pagebreak
Uno de los casos conocidos de éxito de una comunidad de software libre es el estilo de desarrollo de Linus Torvalds en el cual se destaca, lanzar versiones de prueba enseguida y a menudo, delegar cuanto sea posible. La comunidad Linux, parecía semejarse a un gran bazar bullicioso con diferentes agendas y enfoques (adecuadamente reflejado por los depósitos de software Linux, que admitían contribuciones de cualquiera) del cual solo parecía posible que emergiera un sistema coherente y estable mediante una sucesión de milagros. 

\section{¿Quiénes pertenecen a una comunidad?}


Las personas que forman parte de una comunidad de SL pueden ser usuarios, desarrolladores, distribuidores, soportistas, traductores entre otras cosas. Las comunidades pueden abarcar todas estas áreas, o enfocarse en algunas específicas.
\\
\\
Se puede deducir que existe una gran influencia universitaria en el software libre. Esto no es de extrañar, ya que, como se ha podido ver en el capítulo de historia, el software libre –{\bf antes incluso de llevar esta denominación}– ha estado íntimamente ligado a las instituciones educativas superiores. Aún hoy, el verdadero motor del uso y expansión del software libre siguen siendo las universidades y los grupos de usuarios estudiantiles. No es, por tanto, de extrañar que más de un 70\% de los desarrolladores cuenten con una preparación universitaria. El dato tiene mayor importancia si tenemos en cuenta que del 30\% restante muchos no son universitarios porque todavía están en su fase escolar. Aun así, también tienen cabida –{\bf y no por ello son menos apreciados}– desarrolladores que no han accedido nunca a estudios superiores, pero que son amantes de la informática.
\\
\\
Dentro de las comunidades de software libre existe un conjunto de pasos que facilita al simple usuario transformarse en un activo participante del proyecto, tales como: 

\begin{itemize}
     \item Ganar experiencia instalando el software en su computadora o en un servidor Web. 
     \item Contribuir a los foros de discusión.
     \item Contribuir a la documentación y a la promoción.
     \item Reportar errores y verificar las distintas versiones.
     \item Modificar el código para personalizar una operación o corregir un error.
     \item Crear un módulo para extender la funcionalidad.
     \item Entregar parches y módulos para revisión por pares e incorporarlos en el tronco principal del proyecto.
\end{itemize}
%\newpage
Por otro lado, también se ha podido constatar una gran interdisciplinariedad: uno de cada cinco desarrolladores proviene de campos diferentes al de las tecnologías de la información. Esto, unido al hecho de que existe también un número similar de desarrolladores no universitarios, refleja la existencia de una gran riqueza en cuanto a intereses, procedencias y, en definitiva, a la composición de los equipos de desarrollo. Es muy difícil encontrar una industria moderna donde el grado de heterogeneidad sea tan grande como el que se puede ver en el software libre, si es que existe.
\\
\\
En cualquier caso, dentro de una comunidad la mayoría quiere aprender y desarrollar nuevas habilidades ({\bf acerca de un 80\%}) y que muchos lo hacen para compartir conocimientos y habilidades ({\bf 50\%}) o para participar en una nueva forma de cooperación ({\bf alrededor de un tercio}). El primer dato no parece nada sorprendente, habida cuenta de que un profesional con mayores conocimientos se encuentra más cotizado que uno que no los posee. El segundo dato, sin embargo, no es tan fácil de explicar e incluso parece ir en contra de la afirmación de Nikolai Bezroukov, que viene a decir que los líderes de los proyectos de software libre tienen a buena cuenta no compartir toda la información de la que poseen para perpetuar su poder. Mientras tanto, la tercera opción más frecuente es, sin lugar a dudas, fiel reflejo de que los propios desarrolladores se muestran entusiasmados por la forma en la que generalmente se crea el software libre; es difícil encontrar una industria en la que un grupo de voluntarios levemente organizados pueda plantar cara tecnológicamente a las grandes compañías del sector.
\\
\\
La gente que dedica su tiempo al software libre en líneas generales podemos afirmar que el desarrollador de software libre es un varón joven con estudios universitarios ({\bf o en vías de conseguirlos}). La relación del mundo del software libre con la universidad ({\bf estudiantes y profesores}) es muy estrecha, aunque sigue predominando el desarrollador que no tiene que ver nada con el ámbito académico. En cuanto a la dedicación en número de horas, se ha mostrado cómo existe una gran desigualdad al estilo de la postulada en la ley de Pareto. 
Las motivaciones de los desarrolladores –{\bf según ellos mismos}– lejos de ser monetarias y egocéntricas, tal y como suelen asumir economistas y psicólogos, está más bien centrada en compartir y aprender. \emph{Para finalizar, se puede decir que los personajes del mundo del software libre, en el cual la reputación en la gran comunidad del software libre suele depender de más razones que solamente de la codificación de una aplicación libre exitosa}.
\\
\\
